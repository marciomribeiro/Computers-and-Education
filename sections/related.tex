\section{Related Work}

Reducing the dropout rate in programming courses has been achieved in previous work~\cite{}. The authors performed a study during four semesters. They identified three main factors to reduce the dropout rate: (i) using Python as the first introductory programming language, which, according to the authors, may lead the students to better focus on algorithms and problem solving, instead of spending time with advanced concepts of other existing languages at an early stage of the learning process; (ii) using visualization environments to support students and improve the way they can understand abstract concepts related to programming; and (iii) assigning individual problems to each student, hindering students sharing or borrowing solutions to their friends. Like our work, the authors are concerned with the high rate of dropout students. However, while they focus on a solution to decrease such numbers, we provide a strategy---evaluated in seven semesters---to early (30 days) identify potential dropout students. Then, professors and assistants may act to help these students with additional classes and particular conversations. After studying and understanding why exactly they tend to dropout, we then are ready to provide a solution. \todo{achei esse final meio estranho. Nao vendi bem!}