\section{Related Work}

\label{sec:related}

Previous studies introduce aptitude tests to check whether a student will succeed or fail programming courses. To do so, some studies rely on past academic achievements. For example, Butcher et al.~\cite{butcher-predictor-high-school-1985} use high school data and ACT scores to predict college performance in computer science courses. However, their goal is to predict students who had successfully completed one year of study in computer science in general and not only in computer programming. Besides, they use data that are only available at the United States, then it is difficult to replicate to other countries. Similarly, Hostetler~\cite{hostetler-aptitude-1983} uses the college grade point average and math background to perform the predictions.

Another category of predictors rely on surveys to better understand the student skills. Hughes et al.~\cite{ibm-aptitude-test} introduces a survey based on many math and logical questions. Simon et al.~\cite{simon-predictors-ace2006} describe a multi-institutional study to predict success in introductory programming courses. To do so, they set a number of diagnostic tasks (such as map sketching and phone book searching) and perform a qualitative analysis based on short interviews. Dehnadi et al.~\cite{camel-2006} observe the mental models that students use when reasoning about simple assignments and sequence of assignments.

Although the surveys present promising results, applying and replicating these studies is difficult, time consuming, and error prone, specially when considering large class sizes. Differently, the strengths of our study include an automatic predictor of students candidates to fail. Because it is automatic, this reduces significantly the effort of professors interested in predicting such candidates. Despite achieving 72\% of precision, we focus only on one programming language and we use the same professor in all 7 courses at the same university, differently from Simon et al.~\cite{simon-predictors-ace2006} that use data from 11 institutions from three countries.

Lorenzen et al.~\cite{lorenzenC06-mastermind-predictor-sigcse2008} presents a game-based predictor. They claim it is possible to predict failing students by only making them to play the game (named MasterMind). Unfortunately, the given webpage in the paper is not available anymore, so it is hard to set and replicate the study to better support the claims of the paper. Besides, the game does not give any clues to professors and mentors regarding the deficiencies of each student. Online judges, on the other hand, can help on this task, since it is possible to map particular subjects from the curriculum to each problem available in the judge. So, professors can identify that a particular set of students have difficulties in loops, for instance.

Another predictor consists of introducing an assembly-based language designed in a such way that students do not need to have any programming background~\cite{harris-assembly-jcsc2014}. This study correlates the results of tasks implemented in such a language with the midterm exam. They validate their study with 23 students and find a good correlation between the tasks and the midterm exam. In contrast, our evaluation focuses on the first 30 days (\semesterPercentage of the course) and use 7 semesters, totalling \totalStudents students. Therefore, we are able to predict earlier, giving better chances for professors and mentors to recover the students. Like the MasterMind game~\cite{lorenzenC06-mastermind-predictor-sigcse2008}, the given webpage in the paper is no longer available (despite being a very recent work), which makes the replication of the study more difficult.

Semi-automatic approaches (behavior)~\cite{rodrigo-behavioral-ITiCSE2009}

~\cite{emily-icer-2011} midterm exam. The derived models could not predict the students.

~\cite{emily-up-2008} moderate correlation (midterm, nao foca em 30 dias. Nao foca em quem realmente vai perder).

~\cite{susan-sigce2005} questionario.

Usamos online judge systems. BlueJ somente no lab.~\cite{watson-icalt-2013}. Nao foca nos 30 dias.

~\cite{diane-acii-2011} midterm

~\cite{marques-ia-2013} data mining.

%MasterMind predictor (game)~\cite{lorenzenC06-mastermind-predictor-sigcse2008} This paper claims that it is possible to predict weather a student will fail by playing a mastermind game. However, unfortunately the paper presents no data to support this claim. Furthermore, there is no evidence in the literature that the test was successfully applied in a different context.

%Reducing the failure rate in programming courses has been achieved in previous work~\cite{}. The authors performed a study during four semesters. They identified three main factors to reduce the failure rate: (i) using Python as the first introductory programming language, which, according to the authors, may lead the students to better focus on algorithms and problem solving, instead of spending time with advanced concepts of other existing languages at an early stage of the learning process; (ii) using visualization environments to support students and improve the way they can understand abstract concepts related to programming; and (iii) assigning individual problems to each student, hindering students sharing or borrowing solutions to their friends. Like our work, the authors are concerned with the high rate of failure students. However, while they focus on a solution to decrease such numbers, we provide a strategy---evaluated in seven semesters---to early (30 days) identify potential failing students. Then, professors and mentors may act to help these students with additional classes and particular conversations. After studying and understanding why exactly they tend to fail, we then are ready to provide a solution. \todo{achei esse final meio estranho. Nao vendi bem!}