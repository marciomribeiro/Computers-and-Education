\section{Introduction}

The high rate of failing students in introductory programming courses is a problem~\cite{}. In particular, this problem has been reported in several universities around the world~\cite{}. Due to such high failing rates, these courses are often perceived by the students as problematic~\cite{yadin-inroads-acm-11}. In this context, by analyzing several reasons of why the students fail~\cite{why-dropout-icer06}, such as lack of motivation, lack of time, and the chosen programming language, previous work report approaches that reduce these rates~\cite{yadin-inroads-acm-11, xxx}.

These studies focus on data, reasons, and characteristics \textit{after} the student fail. Nevertheless, there is a lack of an approach capable of identifying that students are not comfortable still during the course. This way, professors and assistants would be able to act and consequently help them. However, identifying potential failing students during introductory programming courses is a non-trivial task~\cite{}, specially if we consider that this identification must be done as far as possible, otherwise there will be no enough time to act and the student will drop out the course anyway. When considering programming courses, the situation gets worse even when acting just a bit late, due to the strong prerequisites of understanding previous classes to understand the current one (e.g., to understand loops, students must understand conditional structures).

To minimize this problem, in this article we propose a strategy to \textit{early} identify failing students in introductory programming courses. Our strategy consists of three simple steps. The first one is to make students use a online exercise system. Then, we collect metrics of each student by using such a system. Finally, we execute a clustering algorithm to form groups so that we are able to separate the potential failing students from the other ones. Here, we apply our strategy considering the first 30 days of the programming course.

To evaluate our strategy, we conduct an empirical study regarding 7 courses (3.5 years) with, in total, \totalStudents students. The programming courses have been ministered at the Federal University of Alagoas, Brazil. The results suggest that our strategy can early detect the majority of the failing students within only 30 days. In particular, from the group of students our strategy points as ``likely to fail,'' 72\% of the students indeed fail with 95\% standard confidence level. Moreover, the 28\% our strategy misses is still important to take into account, since at least 33\% of these students have difficulties to pass and reach the final exam. So, although they pass, this set has good candidates that also need special attention. In case professors and assistants help these students, they may avoid final exams and achieve better grades at the end of the course.

In summary, this article provides the following contributions:

\begin{itemize}

	\item A strategy to \textit{early} identify failing students in introductory programming courses;
	
	\item An empirical study assessing the potential of our strategy. We evaluate our strategy by using \totalStudents students from 7 courses during 3.5 years, demonstrating significant potential.

\end{itemize}