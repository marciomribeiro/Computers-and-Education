\section{Introduction}

\begin{itemize}
	
	\item Evaluation: metrics collection during 3.5 years of real programming courses at UFAL, Brazil. Main results: in the first 30 days of the courses we analyzed, our approach could identify students that did not complete the course successfully. We believe that providing early support to these students right after this identification in 30 days, they will have better chance of completing the course.

\end{itemize}

Context: learning during programming courses \todots

Problem: \textit{early} identification of potential failing students \todots

To minimize this problem, in this paper we propose a strategy to \textit{early} identify failing students in introductory programming courses. Our strategy consists of three simple steps. The first one is to make students use a online exercise system. Then, we collect metrics of each student by using such a system. Finally, we execute a clustering algorithm to form groups so that we are able to separate the potential failing students from the other ones. Here, we apply our strategy considering the first 30 days of the programming course.

To evaluate our strategy, we conduct an empirical study regarding 7 courses (3.5 years) with, in total, \totalStudents students. The programming courses have been ministered at the Federal University of Alagoas, Brazil. The results suggest that our strategy can early detect the majority of the failing students within only 30 days. In particular, our strategy successfully identifies 72\% of the failing students with a 95\% confidence level.

In summary, this paper provides the following contributions:

\begin{itemize}

	\item An strategy to \textit{early} identify failing students in introductory programming courses;
	
	\item An empirical study assessing the potential of our strategy. We evaluate our strategy by using \totalStudents from 7 courses during 3.5 years, demonstrating significant potential.

\end{itemize}