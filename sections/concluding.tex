\section{Concluding Remarks}

\label{sec:conclusion}

This article presented a strategy able to early predict potential failing students in introductory programming courses. The strategy is automatic, reducing effort and allowing professors to use it in their courses. We focused on the identification, rather than on approaches to avoid the failings, e.g., changing the programming language. Our strategy consists of three simple steps: the use of an online judge system, the collection of metrics from this system; and the application of a clustering algorithm. In an empirical study using 7 courses (representing 3.5 years) with freshmen students, the results suggest that our strategy can early detect (within only 30 days) the failing students, which is a promising result. In particular, from the group of students our strategy points as ``likely to fail,'' 72\% of the students indeed fail with 95\% standard confidence level. We also found that the remaining set of 28\% of students that passed is still interesting: at least 33\% of them had difficulties to pass and reached the final exam. This way, some of these students need help as well. In case this happens, they may avoid final exams and achieve better grades at the end of the course. We also concluded that the efficacy of our strategy depends on the number of students enrolled in the course. In case the class size is short (e.g., 10 students), the strategy adds little, since the own professor can identify the failing candidates.

As future work, we intend to use other metrics and clustering algorithms to better define and analyze our strategy. Additionally, by using the results of our strategy during actual semesters, we intend to make professors and mentors aware of the potential failing students. Then, we are ready to evaluate whether the strategy application can somehow help on reducing the high rate of failing students in introductory programming courses.