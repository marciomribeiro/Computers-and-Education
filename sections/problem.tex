\section{Problem}

\label{sec:problem}

Due to the high number of students, professors and assistants usually do not identify potential failing students during the courses they teach. In this context, professors are not aware of what actually hinders particular students during the learning process. With no additional help, students have no enthusiasm regarding the classes and get frustrated and disappointed, specially when they see their friends excited about the course. This situation causes shame and timidity, which may lead them to not ask questions or not participate in class. When considering introductory programming courses, this frustration commonly lead students to fail the course~\cite{}.

%Why the problem is relevant, important?
In this context, there is no easy way to \textit{early} identify potential failing students during introductory programming courses. When professors and assistants do not early identify students that tend to fail the programming course, they may act too late---or even may not act, since there is no available time anymore---and the students drop out the course anyway. In case they act a bit late, they still face the hard task of recovering such students, which happens to be even harder in programming courses, where to understand the next class there is a strong prerequisite of understanding the previous ones (e.g., to understand loops, students must understand conditional structures).

This way, not identifying failing students early is a critical problem, specially when we consider that programming is one of the first courses that students face in their computer science university program. If these courses have as high failure rates as claimed, they could be one of the factors influencing the declining number of students taking a degree in computer science \cite{bennedsen-sigcse}.

To minimize the lack of an early identification of potential failing students that would enable professors and assistants to act faster in order to avoid such failings, we next present a strategy that consists of three simple steps: the use of a online judge system, metrics collection, and execution of a clustering algorithm.