\section{Problem}

\label{sec:problem}

The task of identifying the skills and deficiencies of each student in particular is definitely not easy. This happens specially when professors have lots of students with different backgrounds.
%Why this is relevant, important?
Since it is difficult to identify these students, professors are not aware of what actually hinder them during the learning process. Consequently, students get frustrated and disappointed---specially when they see their friends excited about the course---causing shame and timidity, which may lead them to, among other things, not ask questions or not participate in class. When considering introductory programming courses, this frustration commonly lead students to fail the course~\cite{}.

%This situation also happens when considering introductory programming courses.

In this context, there is no easy way to \textit{early} identify potential dropout students during introductory programming courses. When not early identifying students which tend to drop out the programming course, professors and assistants may act late (or even may not act, since there is no time anymore) and the students drop out the course anyway. In addition, \todo{consider more consequences! Why this is painful?!}

%, which would be useful to act somehow to avoid students dropping out of the programming course and even of computer science ones. 

The early identification critical when we consider that programming is one of the courses that students encounter first in their computer science university program. If these courses have as high failure rates as claimed, or if that rumor gets around to students, it could be one of the factors influencing the declining number of students taking a degree in computer science \cite{bennedsen-sigcse}. 