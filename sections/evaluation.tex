\section{Evaluation}

\label{sec:evaluation}

To evaluate our strategy, in this section we present the empirical study we conduct. The objective of this study is to evaluate to what extent our strategy is capable of identifying potential failing students in only 30 days. To explain our evaluation design, we first present the general settings in Section~\ref{sec:participants}. Then, we detail the procedure we use during the evaluation in Section~\ref{sec:procedure}.

%Before explaining our study, we first introduce the terminology we use throughout this article. In particular, we define in what follows four categories: abort, skip, fail, and pass.
%
%\begin{itemize}
%
%	\item \textbf{Abort:} students that aborted the course before the final exam;
%	\item \textbf{Skip:} students allowed to do the final exam but did not show up;
%	\item \textbf{Fail:} students who failed the course;
%	\item \textbf{Pass:} students who successfully passed the course.
%
%\end{itemize}

%Now, we present the objetives and hypothesis (Section~\ref{sec:hypotheses}) of our study, then the participants and material (Section~\ref{sec:participants}) we use, and finally we detail the procedure (Section~\ref{sec:procedure}) we use during the evaluation.

%\subsection{Objective and Hypothesis}

%\label{sec:hypotheses}

%The objective of this study is to evaluate to what extent our strategy is capable of identifying potential failing students. This way, based on this objective, our hypothesis is the following: In the first 30 days, students with lower number of submissions and correct submissions tend to fail the course.

\subsection{Settings}

\label{sec:participants}

The participants of our study are freshmen students of introductory programming courses at the Federal University of Alagoas in Brazil. We collected the metrics we detail in Section~\ref{sec:metrics} of each student by using Huxley. Table~\ref{tab:participants} distributes the number of participants per semester.

\begin{table}[h]
\centering
\begin{tabular}{|c|c|}
\hline
\textbf{Course} & \textbf{Number of enrolled students}\\ \hline
2010.02 & 32\\ \hline
2011.01 & 38\\ \hline
2011.02 & 35\\ \hline
2012.01 & 34\\ \hline
2012.02 & 29\\ \hline
2013.01 & 28\\ \hline
2013.02 & 31\\ \hline
\textbf{TOTAL:} & \totalStudents\\ \hline
\end{tabular}
\caption{Participants per course.}
\label{tab:participants}
\end{table}

The same professor (Paes, one of the authors) taught the 7 semesters. The classes happened in a lab and some exercises available in Huxley were solved during the class. The professor strongly encouraged all students to submit solutions for the problems available in Huxley during extra-class activities, once there is absolutely no penalty in case of wrong submitted solutions. However, during the first 30 days, the use of Huxley was mandatory for approximately 6 exercises (used as a small percentage of the final grade) and for the first formal exam (held closely to the 30th day).

The programming language was C and the program curriculum for the first 30 days was the following: \textit{introduction to algorithms; hello world in C; variables; identifiers; constants; basic structure of a C program; input/output; operators; data types; assignments; and conditional structures}.

%Although Huxley contains more than 300 programming exercises, in our study we rely only on the exercises based on the aforementioned curriculum. We do not expect a student solving problems that would require more advanced subjects, such as arrays.

%The material of this study consists of almost 300 programming exercises in Huxley. They were available for all students of all courses.

%\todo{Descrever: quantas submissoes; quantos problemas? focar somente nos 30 dias?}

%\todo{O que foi feito nos cursos? Aulas expositivas? Listas de exercicios? Importante para qualquer outro professor replicar.}

\subsection{Procedure}

\label{sec:procedure}

Figure~\ref{fig:procedure} illustrates how we perform our evaluation. The result of executing our strategy consists of groups of students according to the clustering algorithm. Now, according to the groups, we have the potential failing students (see the left-hand side in Figure~\ref{fig:procedure}). These students are the ones that have a small number of submissions and correct submissions (represented by circles). Notice we also have students that are potential candidates to successfully pass (represented by stars): due to the high number of submissions to Huxley after 30 days, they seem to be practicing programming really hard. We represent the inconclusive group by squares.

\begin{figure}[htb]
\centering
\includegraphics[width=1.0\textwidth,natwidth=950,natheight=394]{images/Procedure.pdf}
\caption{Checking the strategy results against the actual grades.}
\label{fig:procedure}
\end{figure}

After applying the strategy, we now need to check whether it correctly predicts the failing students after 30 days. To do so, we use the academic system of the Federal University of Alagoas to look for grades and check whether the students failed or not. For example, for the detached circles, the strategy successfully identified that, after 30 days, two students would not pass and they indeed did not. Notice, however, that we may face false positives. Despite indicating the student \textit{John Smith}\footnote{All names we consider in this article are fictitious.} as a failing one after 30 days, such a student seemed to improve herself during the semester and she has been approved. In addition, we may also have false negatives. For instance, our strategy does not point the students \textit{Emma Wilson} and \textit{David Miller} as failing ones. Although \textit{Emma Wilson} seems to be one of the best students after 30 days, he failed the course. The several reasons why this happened is out of the scope of this article.

To better structure and analyze our results and, at the same time, take false positives and false negatives into account, we consider the following two metrics: precision and recall. Precision is the fraction of retrieved students that are relevant, i.e., the students pointed by the strategy that indeed failed the course. We have a perfect precision, 1.0, when every student retrieved by the strategy is relevant (i.e., a failing student), which means we have no false positives. Precision focuses on \textit{quality} and \textit{accuracy}. However, the precision says nothing about whether \textit{all} relevant students were indeed retrieved.

\vspace{0.1cm}
$$
Precision = \frac{| \{relevant\_students\} \cap \{retrieved\_students\} |}{| \{retrieved\_students\} |}
$$
\vspace{0.1cm}

Recall, in its turn, is the fraction of relevant students that are retrieved, i.e., it is the probability of retrieving a failing student. A perfect recall, 1.0, means that we retrieve all failing students, which means we have no false negatives. In this context, recall focuses on \textit{completeness} and \textit{quantity}. Notice that recall says nothing about how many irrelevant students (students that will pass) the strategy retrieved.

\vspace{0.1cm}
$$
Recall = \frac{| \{relevant\_students\} \cap \{retrieved\_students\} |}{| \{relevant\_students\} |}
$$
\vspace{0.1cm}

To better explain these metrics, consider the detached students in Figure~\ref{fig:procedure} as our set (three circles, one square, and one star). The relevant set is \textit{\{Emma, David, Mary, Peter\}}, whereas the retrieved set is \textit{\{John, Mary, Peter\}}. The intersection set is \textit{\{Mary, Peter\}}. This way, we have

\vspace{0.2cm}
\noindent
\scriptsize
\begin{minipage}{.5\linewidth}
\centering
$$
Precision = \frac{|\{Mary, Peter\}|}{|\{John, Mary, Peter\}|} = 67\%;
$$
\end{minipage}
\begin{minipage}{.5\linewidth}
$$
Recall = \frac{|\{Mary, Peter\}|}{|\{\textit{\text{Emma}}, \textit{\text{David}}, Mary, Peter\}|} = 50\%.
$$
\end{minipage}
\normalsize
\vspace{0.2cm}

Our strategy pointed two out of three students as failed ones and they indeed failed. Therefore, we have 67\% of accuracy when identifying potential failing students, raising one false positive. On the other hand, only two out of four students have been pointed as failed ones. This means that the strategy was not able to identify all failing students, raising two false negatives.

Last but not least, after confronting the strategy results with the academic system and summarizing precision and recall, we apply a statistical test to check for significance. Here, we rely on the proportion statistical test based on the Bernoulli distribution~\cite{linear-models-1989} so that we have a binary distribution: fail or pass. In this article, we follow the convention of considering a factor as being significant to the response variable when \textit{p-value} $< 0.05$~\cite{box-statistics-for-experimenters}.

\section{Results and Discussion}

\label{sec:results}

In this section, we describe the results and test our hypotheses before discussing their implications. All data, materials, and R scripts are available at \url{http://www.ic.ufal.br/}.

\subsection{Results}

In our evaluation, we use data of 7 courses (3.5 years) totalling \totalStudents students. We apply our strategy by setting k-means to compute two and three groups. We now proceed separately, reporting the results considering both cases.

\subsubsection{Two groups}

When considering two groups, we set the strategy to consider all students in two categories: fail or pass. Figure~\ref{fig:2-groups} illustrates the results for all courses. Notice that Figure~\ref{fig:2-2011-01} contains an outlier. In this particular case, the strategy pointed that all students but one would fail the course. Clearly this result is a consequence of such outlier. This way, the presence of one outlier might represent a problem when considering two groups.

\begin{figure*}[ht]
     \begin{center}
         \subfigure[2010.02] {
             \includegraphics[scale=0.21,natwidth=480,natheight=480]{images/2-2010-02.png}
             \label{fig:2-2010-02}
         }
         \subfigure[2011.01] {
             \includegraphics[scale=0.21,natwidth=480,natheight=480]{images/2-2011-01.png}
             \label{fig:2-2011-01}
         }
         \subfigure[2011.02] {
             \includegraphics[scale=0.21,natwidth=480,natheight=480]{images/2-2011-02.png}
             \label{fig:2-2011-02}
         }
         \subfigure[2012.01] {
             \includegraphics[scale=0.21,natwidth=480,natheight=480]{images/2-2012-01.png}
             \label{fig:2-2012-01}
         }
         \subfigure[2012.02] {
             \includegraphics[scale=0.21,natwidth=480,natheight=480]{images/2-2012-02.png}
             \label{fig:2-2012-02}
         }
         \subfigure[2013.01] {
             \includegraphics[scale=0.21,natwidth=480,natheight=480]{images/2-2013-01.png}
             \label{fig:2-2013-01}
         }
         \subfigure[2013.02] {
             \includegraphics[scale=0.21,natwidth=480,natheight=480]{images/2-2013-02.png}
             \label{fig:2-2013-02}
         }
     \end{center}
     \caption{Strategy applied with two groups.}
	 \label{fig:2-groups}
\end{figure*}

To better analyze our results, we remove this outlier and execute our strategy again. By using two groups and properly removing the outlier, we achieve the following results for precision and recall:

\vspace{0.2cm}
\noindent
\begin{minipage}{.5\linewidth}
\centering
$$
Precision = \frac{92}{145} = 63\%;
$$
\end{minipage}
\begin{minipage}{.5\linewidth}
$$
Recall = \frac{92}{115} = 80\%.
$$
\end{minipage}
\vspace{0.2cm}

Here we observe a high recall, i.e., 80\%. This means we only miss 20\% of the failing students. We have few false negatives, but this result says nothing about how many false positives (irrelevant students) we retrieved. However, notice that this result (80\%) represents our particular sample. We now need to find the population proportion, enabling us to generalize our findings. To do so, we need to check for statistical significance by executing the proportion hypothesis test. In this context, the test reveals that the population recall is 73\% with a confidence level of 95\% (\textit{p-value} $= 0.045$). Regarding precision, our results show a sample precision of 63\%. To generalize, we again execute the test and find that the population precision would be 56\% (\textit{p-value} $= 0.035$).

\subsubsection{Three groups}

Analogously, we set our strategy to consider three groups. Here, besides the failing and passing groups, there is one extra group where the strategy cannot conclude anything about it. Figure~\ref{fig:3-groups} shows the results for three groups. Differently from two groups, here one outlier does not completely compromise the strategy.

\begin{figure*}[ht]
     \begin{center}
         \subfigure[2010.02] {
             \includegraphics[scale=0.21,natwidth=480,natheight=480]{images/3-2010-02.png}
             \label{fig:3-2010-02}
         }
         \subfigure[2011.01] {
             \includegraphics[scale=0.21,natwidth=480,natheight=480]{images/3-2011-01.png}
             \label{fig:3-2011-01}
         }\\
         \subfigure[2011.02] {
             \includegraphics[scale=0.21,natwidth=480,natheight=480]{images/3-2011-02.png}
             \label{fig:3-2011-02}
         }
         \subfigure[2012.01] {
             \includegraphics[scale=0.21,natwidth=480,natheight=480]{images/3-2012-01.png}
             \label{fig:3-2012-01}
         }\\
         \subfigure[2012.02] {
             \includegraphics[scale=0.21,natwidth=480,natheight=480]{images/3-2012-02.png}
             \label{fig:3-2012-02}
         }
         \subfigure[2013.01] {
             \includegraphics[scale=0.21,natwidth=480,natheight=480]{images/3-2013-01.png}
             \label{fig:3-2013-01}
         }\\
         \subfigure[2013.02] {
             \includegraphics[scale=0.21,natwidth=480,natheight=480]{images/3-2013-02.png}
             \label{fig:3-2013-02}
         }
     \end{center}
     \caption{Strategy applied with three groups.}
	 \label{fig:3-groups}
\end{figure*}

We present the precision and recall for three groups in what follows:

\vspace{0.2cm}
\noindent
\begin{minipage}{.5\linewidth}
\centering
$$
Precision = \frac{73}{91} = 80\%;
$$
\end{minipage}
\begin{minipage}{.5\linewidth}
$$
Recall = \frac{73}{115} = 63\%.
$$
\end{minipage}
\vspace{0.2cm}

Now the strategy returns a precision of 80\% for the sample proportion. To generalize our results, we again execute the proportion hypothesis test. Regarding the population, our results reveal that the precision is 72\% with a confidence level of 95\%. In other words, from the set we retrieve,  we can identify with statistical significance (\textit{p-value} $= 0.04$) 72\% of the students that indeed will fail the course after 30 days. We repeat this process for recall and find 63\% for the sample and 55\% for the population, \textit{p-value} $= 0.033$.

Notice that with three groups the results of precision and recall are inverted when compared to two groups. Our strategy uses k-means, which takes into account two metrics---number of submissions and number of correct submissions---and the number of groups. Both executions are totally independent and the algorithm is not aware of precision, recall, and the academic system. Therefore, we take these results as a coincidence.

\subsection{Discussion}

In this section we discuss the results. We first discuss the number of groups we should set when using our strategy and then we discuss our achievements regarding final exams.

\subsubsection{Two or Three Groups?}

When applying our strategy considering two groups, the results indicate a higher recall when compared to three groups. This means that the strategy can identify the majority of the failing students (raising only few false negatives). Although this is an interesting result, the strategy with two groups yields many false positives. In fact, the precision is lower when compared to three groups, i.e., the set we retrieve contains many students that pass. In this situation, professors and assistants might waste effort trying to recover students that actually do not need recovering. On the other hand, with three groups we have a higher precision, which means professors should give special attention to the students the strategy retrieves, since 72\% of them tend to fail. Nevertheless, since the recall is lower than with two groups, we have more failing students not retrieved by the strategy for three groups.

%we may have there are many failing students not retrieved by the strategy.

In this context, setting the number of groups to execute our strategy seems to depend on the professors priorities and resources. In case the professor has available time and additional assistants to help her, she can probably use two groups, which consists of a more complete retrieved set (the recall is higher for two groups), even though it contains many false positives. However, notice that applying the strategy with two groups is more likely to outliers problems, such the one we depict in Figure~\ref{fig:2-2011-01}.

On the other hand, if the professor wants to avoid false positives because of no available time or few assistants to help, she might prefer to use three groups, despite being aware of potential failing students not retrieved by the strategy (many false negatives, lower recall than with two groups). 

%better precision in exchange for 

\subsubsection{Final Exams}

Even though the strategy pointed students as potential failing ones in the first 30 days, some of them passed, raising false positives. Table~\ref{tab:final-exams} illustrates the precision for two and three groups as well as the false positives. For two and three groups, the precision is 56\% and 72\%, respectively. This way, we have 44\% and 28\% of false positives. As mentioned, the number of false positives is greater when considering two groups.

\begin{table}[h]
\centering
\begin{tabular}{|c|c|c|c|}
\hline
\textbf{Groups} & \textbf{Precision} & \textbf{False Positives} & \textbf{Final Exam}\\ \hline
2 & 56\% & 44\% (53 students) & 35\% (19 out of 53 students)\\ \hline
3 & 72\% & 28\% (18 students) & 33\% (06 out of 18 students)\\ \hline
\end{tabular}
\caption{False positives (students pointed as potential failing ones but passed) that performed the final exam.}
\label{tab:final-exams}
\end{table}

To better understand these false positives, we now analyze their grades at the academic system. In our study, we find that some of these students did not pass in the first place. In fact, they needed to perform the final exam to pass. At the university we focus on this article, the final exam represents a second chance and is only available to students with not enough grades to pass.

This result is important in the sense that, although the strategy might raise false positives, it seems worth to follow these students as well. We find that at least 33\% of them need help (regardless of the number of groups, two or three), otherwise they reach the final exams. This way, professors and assistants can also give special attention to these students, which may improve their learning process, avoid final exams, and lead to better grades.

\subsubsection{Back to our objective}

As mentioned, the objective of our study is ``to evaluate to what extent our strategy is capable of identifying potential failing students in only 30 days.'' Given the results we achieve, we believe our strategy is indeed capable of identifying the failing students early. In particular, from the group of students our strategy points as ``likely to fail,'' 72\% of the students indeed fail with 95\% standard confidence level by using three groups. 

%Also, we notice that the 28\% remaining contains at least 33\% of students that somehow need help, since they reach the final exam.

%\subsubsection{Dropout rate}

%\todo{Discuss how the strategy can minimize the dropout rate. }

\subsubsection{Number of students}

The efficacy of our strategy can strongly depend on the number of enrolled students in the course. In this context, if the number of students is small (e.g., 10 students), the own professor can identify---also early---the students that need help. This scenario is even more promising when the professor counts on assistants to help. Therefore, here our strategy adds little and professors might not notice the benefits of using it.

On the other hand, despite having courses throughout Brazil with few students, it is common to find courses with 30, 40, 50, or even 60 enrolled students. In USA, for instance, this number can be even larger at some universities. In this context, previous studies~\cite{bennedsen-sigcse-failure-rates-2007} considering 63 institutions report introductory programming courses with more than 600 students! The average course size is 116, but only 23\% of the courses have less than 30 students~\cite{bennedsen-sigcse-failure-rates-2007}. In addition, they found that small classes (with less than 30 students) perform better regarding pass rates than larger ones.

When considering larger classes, the task of identifying potential failing students becomes more difficult, time consuming, and error prone. Therefore, an automatic approach like our strategy can better support professors and their assistants. Moreover, once professors identify such students earlier, they may have better chances of helping the students before it is too late. So, the application of our strategy can also contribute to the dropout rate reduction.

%\todo{It seems like the small classes (less than 30 students) do better than the larger ones---the average pass rate in the small classes is 82\% whereas large classes only have an average pass rate of 69\%~\cite{bulletin2007}.}

%\todo{The size of the courses also varies a lot, from 8 students to 645. The mean course size is 116, but 23\% of the courses have less than 30 students~\cite{bulletin2007}.}

\subsubsection{Applying the Strategy versus Applying Exams}

One might wonder what is the difference between applying our strategy and applying a formal exam after 30 days. In this case, students with bad grades are candidates to fail the course. However, notice that one-day exams represent one specific chance to the student. So, it is not uncommon to confirm that this kind of exam often does not truly evaluate the students knowledge/skills. In this context, applying many exams can minimize such a problem. Nevertheless, when considering the professor perspective, grading programming many exams and reasoning at the same time about the potential failing students are time-consuming tasks, increasing effort.

In contrast, by using online judges, professors might evaluate the students continuously. This way, they might better identify the students deficiencies automatically. In addition, online judges allow students to solve many problems (either as ordinary exercises or as formal exams), instead of only a couple commonly found in one-day exams. Here, many exams do not necessarily impact negatively on the professors effort.

Our strategy represents a feasible way to identify potential failing students early. We perform the identification in an automatic way, reducing effort of professors and assistants. In addition, our results reveal that the strategy can identify such students with good precision, minimizing false positives.


%Notice, however, that setting meetings up whenever the strategy identifies potential candidates to fail represents 

%\todo{Posso aplicar muitos mini testes. Mas o esforco vai ser grande. Criar as questoes e corrigir. O uso do online judge ajuda.}

%\todo{Bronca: Estrategia versus Tabela no huxley de scores nos exercicios. Como defender a estrategia? Por que o huxley sozinho nao e suficiente?}

%\begin{itemize}

%	\item So uma prova, nao haveria muito feedback. Dia ruim?
%	\item Varios minitestes, ha uma avalia��o constante; a quantidade de dados e feedback e muito maior!

%\end{itemize}

%\subsubsection{O que a estrategia oferece? Por que devo usa-la? Por que o que existe nao e suficiente?}

%\begin{itemize}
%
%	\item Automatico! Diminui o esforco para identificar.
%	\item Boa precisao.
%
%\end{itemize}

\subsection{Threats to validity}

In this section we present the threats to validity of our study. Although our sample is reasonably big (227 students), our study has homogeneities that might pose threats. For example, the same professor for all the 7 courses and the same language used (C language) threats external validity. In this way, it is difficult to extrapolate the results to other contexts. Nevertheless, our strategy uses metrics to identify potential failing students. In this context, we argue that the metrics we use (submissions and correct submissions) do not necessarily depend on factors such as the professors and even less on the adopted languages. However, our claim is not enough and we need further studies to better generalize our conclusions.

The use of Huxley threats internal validity. Students must know how to use the system to submit their solutions. If the students somehow do not get used to the system, they might feel frustrated, hindering their learning and, consequently, biasing our results. We minimize this threat by introducing Huxley in the very first classes as well as by using assistants to help the students on how to use Huxley.