\section{Empirical Evaluation}

\label{sec:evaluation}

To evaluate our strategy, in this section we present the empirical study we conduct. Here, we follow the convention of \todots.

\subsection{Objective and Hypotheses}

The objective of this study is to evaluate to what extent our strategy \todo{dar um nome para a strategy?} is capable of identifying potential dropout students. This way, based on this objective, our hypotheses are the following:

\begin{enumerate}[RH1]

	\item In the first 30 days, students with higher number of submissions and correct submissions tend to pass the course;
	
	\item In the first 30 days, students with lower number of submissions and correct submissions tend to not pass the course.

\end{enumerate}

\subsection{Variables}

\subsection{Material}

The material of this study consists of almost 300 programming exercises in Huxley. 

\subsection{Participants}

The participants of our study consist of students of introductory programming courses at the Federal University of Alagoas, Brazil. We ministered these courses during 3,5 years and collected the results of each student by using Huxley. The professor informed all students that the use of Huxley was mandatory during the courses. Table~\ref{tab:participants} distribute the number of participants per semester.

\begin{table}[h]
\centering
\begin{tabular}{c c}
\hline
\textbf{Course} & \textbf{Number of enrolled students}\\ \hline
2010.02 & 34\\ \hline
2011.01 & x\\ \hline
2011.02 & x\\ \hline
2012.01 & 34\\ \hline
2012.02 & x\\ \hline
2013.01 & x\\ \hline
2013.02 & x\\ \hline
\end{tabular}
\caption{Participants per course.}
\label{tab:participants}
\end{table}

\subsection{Terminology}

To better explain and report the results of our study, in this paper we consider the following terminology:

\begin{itemize}

	\item \textbf{Abort:} the number of students aborting the course before the final exam;
	\item \textbf{Skip:} the number of students not showing up for the final exam, but was allowed to;
	\item \textbf{Fail:} the number of students who failed the course;
	\item \textbf{Pass:} the number of students who passed the course.

\end{itemize}

\subsection{Execution and Deviations}

\section{Results and Discussion}

In this section, we describe the results and test our hypotheses before discussing their implications (All data, materials, and R scripts are available at \url{http://www.ic.ufal.br/}). We now proceed separately, reporting the results where our strategy pointed out students who would fail the course and who would pass the course.



\begin{itemize}

	\item Olhar se, para os alunos onde o algoritmo errou, eles foram pra prova final.
	\item Olhar se, para os alunos onde o algoritmo acertou, quantos deles foram drop out.

\end{itemize}

\subsection{Threats to validity}